%%
%% Author: magnus.silverdal
%% 2019-10-04
%%

% Preamble
\documentclass[11pt]{article}

% Packages
\usepackage{amsmath}
\usepackage{graphicx}
\usepackage{fancyhdr}

%data
\title{We Are Robots}
\author{Magnus Silverdal}
\def\inst{Teknikprogrammet}
\def\typeofdoc{}
\def\course{Programmering 2}
\def\name{Magnus Silverdal}
\def\username{Magnus.Silverdal}
\def\email{\username{}@ga.ntig.se}
\def\graders{Magnus Silverdal}

%Sidhuvud och sidfot
\lfoot{\footnotesize{\name, \\ \email}}
\rfoot{\footnotesize{\today}}
\lhead{\sc\footnotesize\title}
\rhead{\nouppercase{\sc\footnotesize\leftmark}}
\pagestyle{fancy}
\renewcommand{\headrulewidth}{0.2pt}
\renewcommand{\footrulewidth}{0.2pt}

% Document
\begin{document}
    \maketitle
    \newpage
    \section{Beskrivning av uppgiften}
    Er uppgift är att skapa en simulering av en robotvärld. Världen befolkas av två typer av robotar, ljusälskare och ljushatare.
    Robotarna behöver också mat och vatten för att fungera. I världen finns lampor som sprider ljus och mat och vattenstationer
    där robotarna kan fylla på sina energiresurser. Världen har en fix storlek, x*y och kan antingen visualiseras som ASCII-art
    eller med pixelgrafik. (se exemplen från game of life).  Här är ett exempel med ASCII:
    R är robotar, L är ljuskällor, M och V är mat och vatten. Om ni vill kan ni också tillåta väggar (X) i er värld, ni måste
    åtminstone ha väggar runt den så att robotarna inte rymmer.
    \begin{verbatim}
    xxxxxxxxxxxxxxxxxxxx
    x R                x
    x    L   M         x
    x V                x
    x              R   x
    xxxxxxxxxxxxxxxxxxxx
    \end{verbatim}
    I varje steg av simuleringen rör sig robotarna, och tappar lite energi. De rör sig mot/bort från ljus och allt eftersom
    de tappar energi blir de hungrigare och mer villiga att bortse från ljus och mörker för att hitta energi.

    Skapa ett system av klasser med klassdiagram för att genomföra och visualisera simuleringen. Tänk på att alla föremål i
    världen delar en hel del egenskaper och att det kan utnyttjas när ni skapar klassdiagrammet.

    Implementera sedan klassdiagrammet i ett program.

    \section{Genomförande}
    \subsection{Klassdiagram och systembeskriving}
    Första delen i detta projekt blir att analysera problemet och skapa ett klassdiagram för er lösning. Detta gör ni i grupp.
    Prata först igenom problemet och fundera om ni förstått hur allt ska fungera. Använd denna kunskap för att bygga upp ett
    fullständigt klassdiagram över lösningen. Ni behöver också förtydliga vad varje metod i klasserna ska göra i en systembeskrivning.
    Börja med att skissa på tavlan. När ni tror att ni har fått till något som fungerar ska ni koppla upp er prå projektorn
    och föra över klassdiagrammet till DIA.
    \subsection{Implementation och dokumentation}
    Därefter delas ni in i mindre grupper där ni ska implemenetera era klassdiagram. Bygg upp ett projekt i IntelliJ,
    lägg upp det på Github och bygg klass för klass. Se till att ni kan testa klasserna allt eftersom ni bygger upp dem.

\end{document}