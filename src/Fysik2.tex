%%
%% Author: magnus.silverdal
%% 2020-05-04
%%

% Preamble
\documentclass[11pt]{article}

% Packages
\usepackage{amsmath}

% Document
\begin{document}
\section{Kapitel 1}
    \begin{itemize}
    \item{\textbf{Kraftmoment} Momentjämvikt blir ett vilkor för att lösa uppgifter med krafter som vrider.}

    \item{\textbf{Cirulär rörelse} Centripetalaccelerationen $v^{2}/r$ blir villkoret för cirkulär rörelse och används för att analysera en känd cirkulär rörelse. Newtons andra lag ger kraften som upprätthåller rörelsen.}

    \item{\textbf{Kastbana} Genom att lösa ekvationerna från rörelse i x och y-led separat kan kastbanan förstås. Det som kopplar mellan dem är tiden.}
    \end{itemize}
\section{Kapitel 2}
    \begin{itemize}
    \item{\textbf{Harmonisk svängningsrörelser} Kraftsituationen för pendlar och vikter i fjädrar ger en periodisk lösning av formen $y(t) = A \sin{ (\omega t)}$. }

    \item{\textbf{Mekaniska vågor} Ljud och andra mekaniska vågor beter sig på samma sätt som vikter och fjädrar. Vågor reflekteras och interfererar med varandra. En stående våg är en speciell kombination av reflektion och interferens}

    \item{\textbf{Ljud} Speciellt för ljud är ljudstyrkan och dopplereffekten}
    \end{itemize}
\section{Kapitel 3}
\begin{itemize}
    \item{\textbf{Elektrisk fält} Kraften på en laddad partikel beskrivs av ett elektrisk fält.}

    \item{\textbf{Magnetism} Magnetism kan vara en materialegenskap men oftast används magnetism som kommer från ström genom en elektrisk ledare. Det magnetiska fältet förstärks med hjälp av spolar. Laddade partiklar i rörelse påverkas av magnetfält}

    \item{\textbf{Induktion} Förändringar av magnetisk flödestäthet motverkas av induktion. Det innebär att en spänning uppstår som vill driva en ström som motverkar förändringen av det magnetiska fältet. Detta är grundem till växelström}
\end{itemize}

\section{Kapitel 4}
\begin{itemize}
    \item{\textbf{Elektromagnetiska vågor} Detta är periodiska variationer i det elektriska och magnetiska fälten. De följer fysiken i vågrörelseläran så interferens, diffraktion och dopplereffekt gäller även för ljus och andra EM-vågor.}

    \item{\textbf{Svartkroppsstrålning} En speciell form av EM-strålning är svartkroppsstrålning. Det finns en koppling mellan temperatur och utsänt ljus}

    \item{\textbf{Vågrörelselära} Elektromagnetiska vågor kan också beskrivas som partiklar, fotoner, och beroende på hur de växelverkar med omgivningen har de våg eller partikelegenskaper.}

    \item{\textbf{Bohrs atommodel} Elektronen som en stående våg runt atomkärnan förklarar varför energi hos elektronen ser ut somden gör och förklarar också varför atomer sänder ut och absorberar ljus på de sätt de gör.}
\end{itemize}

\end{document}